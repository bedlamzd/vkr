\chapter{ОПИСАНИЕ РАБОТЫ СИСТЕМЫ}
    \begin{figure}[H]
        \centering
        \includegraphics[width=0.5\linewidth]{printer_final}\\
        \caption{Конечный вид прототипа системы автоматического фигурного нанесения пищевой пасты}
    \end{figure}

    Перед эксплуатацией модуля необходимо провести калибровку камеры и сканера в целом с помощью специального паттерна. Калибровка камеры позволяет получить матрицу её внешних и внутренних параметров -- положение камеры в пространстве, фокусное расстояние и реальный центр изображения.
    Калибровка сканера позволяет получить параметры $ H $ и $ \alpha $ по формулам из раздела \ref{sec:scan_calib}. 
    
    \begin{figure}[H]
        \centering
        \includegraphics[width=0.4\linewidth]{camera_calibration}
        \caption{Процесс калибровки камеры}
    \end{figure}
    
    Эксплуатация системы происходит следующим образом. В рабочую зону располагают объекты, закрывают дверцу прототипа и выключается освещение. По команде оператора начинается процесс съёмки видео для обработки. По окончанию съёмки файл отправляется на компьютер, который обрабатывает полученные данные по алгоритму описанному в главе \ref{chap:algorithms}.
    
    \begin{figure}[H]
        \centering
        \includegraphics[width=0.4\textheight]{scan_example}
        \caption{Процесс обработки видео}
    \end{figure}
    \begin{figure}[H]
        \centering
        \includegraphics[width=0.4\textheight]{pointcloud}
        \caption{Облако точек сцены}
    \end{figure}
    \begin{figure}[H]
        \begin{subfigure}{0.5\linewidth}
            \includegraphics[width=\linewidth]{pointcloud_gcode_1}
        \end{subfigure}
        \begin{subfigure}{0.5\linewidth}
            \includegraphics[width=\linewidth]{pointcloud_gcode_2}
        \end{subfigure}
        \caption{Наложение контура из dxf-файла на объект}
    \end{figure}
    
    После обработки данных сгенерированный gcode отправляется обратно в прототип на исполнение.