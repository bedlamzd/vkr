\conclusion
В работе была разработана система 3D-сканирования для прототипа системы автоматического фигурного нанесения пищевой пасты. Разработанная система удовлетворяет требованиям технического задания.

Система строит облако точек рабочей зоны по видео файлу процесса сканирования, находит объекты в рабочей зоне и генерирует gcode на основе заданного dxf файла для нанесения пасты на объекты. Скорость обработки видео файла составляет примерно 10 с. Итоговая стоимость компонентов системы составляет 4850 рублей, что значительно дешевле представленных на рынке аналогов.

Дальнейшим развитием работы будет улучшение качество входных данных и реализация сканирования в цвете. Также улучшение алгоритма генерации gcode и реализация более совершенного алгоритма поиска объектов в облаке. Помимо этого, при условии успеха реализации сканирования в цвете, возможно переработать систему привязки координат, что позволит добиться большей надёжности. Написание методических рекомендаций по выбору метода аппроксимации будет являться отправной точкой при написании пользовательской документации.