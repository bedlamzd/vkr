\intro
В настоящее время существует множество задач, требующих бесконтактного метода измерений: ориентирование в пространстве, измерение объектов, реконструкция объектов, сбор биометрических данных, реверс-инжиниринг, а также дизайн и творчество. Подобных задач с каждым годом становится всё больше, и таким образом растёт важность 3D-сканирования и следовательно необходимость эффективных алгоритмов, решающих конкретную задачу. Кроме того, некоторые задачи требуют уникальных встроенных решений, что представляет собой как правило ещё и конструкторскую задачу.

Одним из конкретных применений такой технологии является фигурное нанесение глазури на различные кондитерские изделия. На данный момент такая операция как правило выполняется вручную, что означает высокую стоимость, длительное производство и низкую повторяемость. Автоматизация этого процесса происходит только на крупных предприятиях, где производят большие партии однотипных кондитерских изделий.

Относительно недавно стали появляться автоматизированные комплексы для нанесения рисунков на различные кондитерские изделия, однако они всё равно ограничены конкретными видами изделий (например, плоские крекеры). Такие комплексы работают по принципу 3D-принтера или ЧПУ-станка. Они печатают в одной плоскости, не имея возможности наносить материал с учётом отклонений формы, не говоря уже об изделиях сложной формы, таких как овсяное печенье, кексы, торты и т.д., у которых форма в общем случае не только не плоская, но и обладает множеством искривлений, выпуклостей и впадин. Данное обстоятельство мешает автоматизации нанесения произвольных рисунков на произвольные изделия.

Эту проблему можно решить, если встроить в печатающую установку 3D-сканер, который строил бы карту поверхностей изделий, учитываемую при печати. [Данная работа посвящена разработке такого модуля 3D-сканирования для кондитерского принтера, разработанного студентом Университета ИТМО. бла бла что-то ещё]