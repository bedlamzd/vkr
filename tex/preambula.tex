%% Essential settings
\documentclass[%
bachelor,    % тип документа
notitlepage, % без титульника
%natbib,     % использовать пакет natbib для "сжатия" цитирований
subf,        % использовать пакет subcaption для вложенной нумерации рисунков
href,        % использовать пакет hyperref для создания гиперссылок
colorlinks,  % цветные гиперссылки
%fixint,     % включить прямые знаки интегралов
]{disser}

\usepackage[
  a4paper, mag=1000,
  left=2.5cm, right=1cm, top=2cm, bottom=2cm, headsep=0.7cm, footskip=1cm
]{geometry}

\usepackage{cmap}

\usepackage[intlimits]{amsmath}
\usepackage{amssymb,amsfonts}
\usepackage{mathtools}
\usepackage{cancel}

\usepackage[T2A]{fontenc}
\usepackage[utf8]{inputenc}
\usepackage[english,russian]{babel}
\ifpdf\usepackage{epstopdf}\fi
\usepackage[autostyle]{csquotes}

% Шрифт Times в тексте как основной
%\usepackage{tempora}
% альтернативный пакет из дистрибутива TeX Live
%\usepackage{cyrtimes}

% Шрифт Times в формулах как основной
%\usepackage[varg,cmbraces,cmintegrals]{newtxmath}
% альтернативный пакет
%\usepackage[subscriptcorrection,nofontinfo]{mtpro2}

\usepackage[%
  style=gost-numeric,
  backend=biber,
  language=auto,
  hyperref=auto,
  autolang=other,
  sorting=none
]{biblatex}

%\addbibresource{thesis.bib}

\usepackage{graphicx}
% Плавающие рисунки "в оборку".
\usepackage{wrapfig}

% Номера страниц снизу и по центру
%\pagestyle{footcenter}
%\chapterpagestyle{footcenter}

% Точка с запятой в качестве разделителя между номерами цитирований
%\setcitestyle{semicolon}

% Использовать полужирное начертание для векторов
\let\vec=\mathbf

% Включать подсекции в оглавление
\setcounter{tocdepth}{2}

% Подписи к рисункам как требует ИТМО
\usepackage[labelsep=endash]{caption}
\addto\captionsrussian{\renewcommand{\figurename}{Рисунок}}

% Названия разделов ПРОПИСНЫМИ буквами
%\renewcommand{\introname}{ВВЕДЕНИЕ}
%\renewcommand{\conclusionname}{ЗАКЛЮЧЕНИЕ}

% Тире в качестве маркера списка


% Code
%\usepackage[gobble=auto]{pythontex}

% Listing environment
\usepackage{minted}
\newenvironment{code}{\captionsetup{type=listing}}{}
\renewcommand{\listingscaption}{Листинг}

\usepackage{lipsum}

% Выделение текста, чтобы не забыть про правки
\usepackage{todonotes}
\newcommand{\hl}[1]{\todo[color=yellow,inline]{#1}}